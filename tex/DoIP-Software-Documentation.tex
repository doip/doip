\documentclass[a4paper]{doipdoc}

\title{DoIP-Software-Documentation}

\begin{document}

\maketitle

\tableofcontents

\section{Introduction}

	This document describes the open source software components for
	DoIP (Diagnostics over IP) which are all hosted in Github.

	They are all located at \url{https://github.com/doip}.

	The implementation is distributed over several repositories.
	
	\begin{description}
		
		\item[\code{doip}:] This is the general repository which contains
		overall content for the whole project. For example this document
		is located in this repository. All issues for the subprojects
		are also managed here. 

		\item[\code{doip-library}:] Contains source files for DoIP
		communication which can be used in a DoIP simulation as well as
		in an diagnostic tester for DoIP.

		\item[\code{doip-simulation}:] Contains source files which are
		specific for a DoIP simulation. The project does not contain an
		executable. It is still a library which can be used in a custom
		DoIP simulation

		\item[\code{doip-custom-simulation}:] This project finally produces
		an executable to start an DoIP simulation. It is supposed to create 
		a copy of the source files and start your own implementation.

		\item[\code{doip-tester}:] This project contains mainly unit tests
		for testing a real ECU or the simulation. The unit tests expect
		that somewhere a DoIP ECU or a simulation is running. The unit tests
		can be executed with gradle.

		\item[\code{doip-logging}:] Contains just a wrapper around 
		log4j2. The intention was to have the possibility to create
		a different impementation than log4j. It is also easier to
		have only one central place which refers to log4j2, so updates 
		to a newer version are easier to manage.

		\item[\code{doip-junit}:] It's a wrapper class for the class
		Assert in junit. The problem in junit is that asserts will be written
		to stderr. That makes it difficult to read logfiles created
		by the build system Gradle. The wrapper class here just logs
		a failed assert also to stdout, so they appear in a logfile at
		the right position.

	\end{description}

\section{Getting Started}

	\subsection{Build and Install Example Simulation}

		An example to run the DoIP simulation is available at 
		\url{https://github.com/doip/doip-custom-simulation.git}.
		To run the DoIP simulation follow the following steps:

		\begin{itemize}
			\item Clone the project \code{doip-custom-simulation}:

			\terminal{git clone https://github.com/doip/doip-custom-simulation.git}

			\item Now the source files of this project are located in the
			folder \code{doip-custom-simulation}. 

			\item Later there might be updates available at Github. To 
			update the repository navigate into the folder 
			\code{doip-custom-simulation} and run git to update the local
			repository

			\terminal{cd doip-custom-simulation \newline
			git pull origin master}

			\item We want to have the possibility to update the example 
			when there are new versions available but we also want
			to create our own simulation with our custom settings. Therefore
			we create a copy of the exampe repositoriy.
			
			\item Create a copy of the repository to a meaningful name (here
				it is \code{doip-myproject-simulation}).

			\terminal{cp doip-custom-simulation doip-myproject-simulation}

			\item Delete the files for git from our copy

			\terminal{rm -rf doip-myproject-simulation/.git/}

			\item Navigate into the folder of our copy

			\terminal{cd doip-myproject-simulation}

			\item Change the name of the project in the file 
			\code{settings.gradle} from
			\code{doip-custom-simulation} to \code{doip-myproject-simulation}

			\item Change version number in file 
			\code{build.gradle}, for example to 1.0.0

			\item Modify the start script because name of JAR file has changed.
			It is located in \code{src/main/dist/start.sh}.

			\item Modify all configuration files in \code{src/main/dist}.

			\item Build the project

			\terminal{./gradlew build}

			\item Install the distribution

			\terminal{./gradlew installDist}

			\item Navigate to the installation folder

			\terminal{cd build/install/doip-myproject-simulation/}

			\item Start the simulation

			\terminal{start.sh gateway.properties}

		\end{itemize}

	\subsection{Configuration of the DoIP Simulation}
		
		The main configuration file is given at a argument to the shell script
		"start.sh". This file is a normal Java property file. The following 
		listing is the example configuration which is used in the repository
		"doip-custom-simulation".

		\lstinputlisting[caption={gateway.properties}]{../submodules/doip-custom-simulation/src/main/dist/gateway.properties}
		% \lstinputlisting[firstline=2, lastline=5, firstnumber=2]{../submodules/doip-custom-simulation/src/main/dist/gateway.properties}

		In that file are references to the configuration files for the ECUs.
		The following listing shows the configuration file for an ECU:

		\lstinputlisting[caption={EMS.properties}]{../submodules/doip-custom-simulation/src/main/dist/EMS.properties}

		Within the ECU property file there is a reference to one or more
		configuration files for the UDS messages. The following listing 
		shows such a configuration file for UDS messages.

		\lstinputlisting[caption={standard.uds}]{../submodules/doip-custom-simulation/src/main/dist/standard.uds}

		In addition there can be so called modifiers added in the UDS configuration
		file. In case a request matches it will send the response and in addition
		it can modify other responses based on the specified request.

		\lstinputlisting[caption={modifier.uds}]{listings/modifier.uds}

\section{Customization of DoIP Simulation}

		The possibilities to configure the DoIP simulations might might not
		cover all use cases which are needed for a DoIP simulation. To
		cover this use case the DoIP simulation was designed to give you
		to possiblity to override the implementation of the predefined
		functions. This will be done by creation of a custom class which
		inherits from an existing class in the DoIP simulation and overrides
		the functions.

		The problem is that the existing DoIP simulation can not create
		an instance of the inherited class, because it is not yet known 
		at implementation time. The solution is to create a custom
		main function where th first instance of a DoIP simulation will 
		be created which will now be the inherited class. 

		This is already done in the project \code{doip-custom-simulation}.
		Create a copy of the source files and adjust the implementation
		according to your needs.
		
		The following listing shows how to create an own main function
		and create a custom DoIP simulation.

		\lstinputlisting[language=Java, caption={doip/custom/simulation/Main.java}]{../submodules/doip-custom-simulation/src/main/java/doip/custom/simulation/Main.java}
		
		\lstinputlisting[language=Java, caption={doip/custom/simulation/CustomGateway.java}]{../submodules/doip-custom-simulation/src/main/java/doip/custom/simulation/CustomGateway.java}

		\lstinputlisting[language=Java, caption={doip/custom/simulation/CustomEcu.java}]{../submodules/doip-custom-simulation/src/main/java/doip/custom/simulation/CustomEcu.java}

\section{Gradle Build Tool}

	All Java projects use Gradle as a build tool. It is not required that
	each user will install Gradle. All projects use the so called 
	"Gradle wrapper" which makes it easy to run and ensure that all users
	are using the same Gradle version.

	The project layout in a Gradle project is defined by convention.

	\begin{description}

		\item[\code{settings.gradle}]: Settings for gradle, for example
		the project name

		\item[\code{build.gradle}]: Build script where is defined what
		to build and how to build it

		\item[\code{src/main/java}]: Source code of the project which will
		be build into one jar file

		\item[\code{src/test/java}]: Unit tests which will be executed after
		compilation
		
		\item[\code{src/main/dist}]: All these files and folders will be packed 
		also into the distribution (only if distribution plugin is used).

	\end{description}

	To run a Gradle task just type \code{./gradlew} in a bash
	followed by the gradle task.

	To build a project just run 

	\terminal{./gradlew build}

	The build task will also execute the unit tests. The comiled jar will be 
	located at \code{build/libs}. If distribution plugin is used then the 
	distribution will be build into folder \code{build/distributions}.

	In case of using the distribution plugin you can run the following command
	to extract the distribution.

	\terminal{./gradlew installDist}

	After that the extracted files can be found in the folder 
	\code{build/install}. You can navigate in the subfolder and run the
	application. In our case there should be a shell script "start.sh" 
	which can be called.


\end{document}

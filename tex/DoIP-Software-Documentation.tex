\documentclass[a4paper]{article}


\usepackage{xcolor}
\usepackage{hyperref}
\usepackage[utf8]{inputenc}

\hypersetup{
	colorlinks=true,
	linkcolor=blue,
	citecolor=blue,
	filecolor=blue,
	urlcolor=blue
}

\newcommand{\terminal}[1]{

\colorbox{gray}{\parbox{\textwidth}{\texttt{\textcolor{white}{\footnotesize{#1}}}}}


}

\title{DoIP-Software-Documentation}

\begin{document}

\maketitle

\tableofcontents

\section{Introduction}

\section{Getting Started}

	\subsection{Build and Install Example Simulation}
		To run the DoIP simulateion follow the following steps:

		\begin{itemize}
			\item Clone the project doip-custom-simulation:

			\terminal{git clone https://github.com/doip/doip-custom-simulation.git}
			
			\item Create a copy of the repository to a meaningful name (here
				it is "doip-myproject-simulation".

			\terminal{cp doip-custom-simulation doip-myproject-simulation}

			\item Delete the files for git from our copy

			\terminal{rm -rf doip-myproject-simulation/.git/}

			\item Navigate into to the folder of our copy

			\terminal{cd doip-myproject-simulation}

			\item Change the name of the project in the file settings.gradle from
			doip-custom-simulation to doip-myproject-simulation

			\item Change version number in file build.gradle, for example to 1.0.0

			\item Build the project

			\terminal{./gradlew build}

			\item Install the distribution

			\terminal{./gradlew installDist}

			\item Navigate to the installation folder

			\terminal{cd build/install/doip-myproject-simulation/}

			\item Start the simulation

			\terminal{start.sh gateway.properties}

		\end{itemize}

	\subsection{Configuration of the DoIP Simulation}
		
		The main configuration file is given at a argument to the shell script
		"start.sh". This file is a normal Java property file. The following 
		listing is the example configuration which is used in the repository
		"doip-custom-simulation".



\section{Software Modules}

\end{document}
